\input{../header.tex}

\begin{document}

%%%%%%%%%%%%%%%%%%%%%%%%%%%%%%%%%%%%%%%%%%%%%%%%%%%
\begin{frame}[fragile] \frametitle{}

\vfill

{\fontsize{0.7cm}{0cm}\selectfont Lecture 09 \\\vspace{0.2cm}
Measuring Airline On-time Performance}\\\vspace{0.5cm}
30 September 2015

\vspace{2cm}

\begin{minipage}{0.6\textwidth}
Taylor B. Arnold \\
Yale Statistics \\
STAT 312/612
\end{minipage}
\hfill
\begin{minipage}{0.3\textwidth}\raggedleft
\includegraphics[scale=0.3]{../yale-logo.png}
\end{minipage}%

\end{frame}

%%%%%%%%%%%%%%%%%%%%%%%%%%%%%%%%%%%%%%%%%%%%%%%%%%%
\begin{frame}[fragile] \frametitle{}

{\color{yaleblue}\fontsize{16pt}{20pt}\selectfont Notes}

\begin{itemize}
\item Problem Set \#2 - Due next class
\item Problem Set \#3 - Due the following Wednesday
\item Midtern I - Two weeks from today
\end{itemize}

\end{frame}

%%%%%%%%%%%%%%%%%%%%%%%%%%%%%%%%%%%%%%%%%%%%%%%%%%%
\begin{frame}[fragile] \frametitle{}

{\color{yaleblue}\fontsize{16pt}{20pt}\selectfont Goals for today}

\begin{itemize}
\item Review of time
\item Simulation of the multivariate F-test
\item Introduction to ASA airline dataset
\end{itemize}

\end{frame}

%%%%%%%%%%%%%%%%%%%%%%%%%%%%%%%%%%%%%%%%%%%%%%%%%%%
\begin{frame}[fragile] \frametitle{}

\begin{flushright}
{\color{yaleblue}\sc\fontsize{1cm}{0cm}\selectfont Review from Last Time}
\end{flushright}

\end{frame}

%%%%%%%%%%%%%%%%%%%%%%%%%%%%%%%%%%%%%%%%%%%%%%%%%%%
\begin{frame}[fragile] \frametitle{}

We did a lot of matrix manipulations in the proofs of these
two results. The most important `big picture' results to
remember are:

\begin{itemize}
\item If $B$ is a symmetric idempotent matrix and
$u \sim \mathcal{N} (0, \mathbb{I}_n)$, then
$u^t B u \sim \chi^2_{\text{tr(B)}}$. \pause
\item If $B$ is a symmetric idempotent matrix, then
all of $B$'s eigenvalues are $0$ or $1$. In terms of
the $Q^t \Lambda Q$ eigen-value decomposition, this
helps explain why we think of $P$ and $M$ as projection
matricies.
\end{itemize}

\end{frame}


%%%%%%%%%%%%%%%%%%%%%%%%%%%%%%%%%%%%%%%%%%%%%%%%%%%
\begin{frame}[fragile] \frametitle{}

The Hypothesis test $H_0: \beta_j = b_j$ yields the
following {\bf T-test}:
\begin{align*}
t &= \frac{\widehat{\beta}_j - b_j}{\sqrt{s^2  \left( (X^t X)^{-1}_{jj} \right)}} \\
&= \frac{\widehat{\beta}_j - b_j}{\text{S.E.}(\widehat{\beta}_j)} \\
&\sim t_{n-p}
\end{align*}

\end{frame}


%%%%%%%%%%%%%%%%%%%%%%%%%%%%%%%%%%%%%%%%%%%%%%%%%%%
\begin{frame}[fragile] \frametitle{}

The Hypothesis test $H_0: D\beta = d$ for a full rank $k$ by $p$ matrix
$D$ yields the following {\bf F-test}:
\begin{align*}
F &= \frac{(\text{SSR}_R -  \text{SSR}_U) / k }{\text{SSR}_R / (n - p)}
\end{align*}
Where we let $\text{SSR}_U$ be the sum of squared residuals of the unrestricted
model ($r^t r$) and $\text{SSR}_R$ be the sum of squared residuals of the
restricted model (where the sum of squares is minimzed subject
to $D\beta = d$).

\end{frame}

%%%%%%%%%%%%%%%%%%%%%%%%%%%%%%%%%%%%%%%%%%%%%%%%%%%
\begin{frame}[fragile] \frametitle{}

\begin{flushright}
{\color{yaleblue}\sc\fontsize{1cm}{0cm}\selectfont F-Test confidence region}
\end{flushright}

\end{frame}

%%%%%%%%%%%%%%%%%%%%%%%%%%%%%%%%%%%%%%%%%%%%%%%%%%%
\begin{frame}[fragile] \frametitle{}

\begin{flushright}
{\color{yaleblue}\sc\fontsize{1cm}{0cm}\selectfont ASA Flight Data}
\end{flushright}

\end{frame}


\end{document}











